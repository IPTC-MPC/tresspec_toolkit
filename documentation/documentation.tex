\documentclass[a4paper, 12pt,
%10pt,								% Schriftgröße (12pt, 11pt (Standard))
%BCOR1cm,							% Bindekorrektur, bspw. 1 cm
%DIVcalc,							% führt die Satzspiegelberechnung neu aus
%											  s. scrguide 2.4
twoside,							% Doppelseiten
openright,            % neue Sections auf rechter Seite beginnen
%twocolumn,						% zweispaltiger Satz
%halfparskip*,				% Absatzformatierung s. scrguide 3.1
%headsepline,					% Trennline zum Seitenkopf	
%footsepline,					% Trennline zum Seitenfuß
%titlepage,						% Titelei auf eigener Seite
%normalheadings,			% Überschriften etwas kleiner (smallheadings)
%idxtotoc,						% Index im Inhaltsverzeichnis
%liststotoc,					% Abb.- und Tab.verzeichnis im Inhalt
%bibtotoc,						% Literaturverzeichnis im Inhalt
%abstracton,					% Überschrift über der Zusammenfassung an	
%leqno,   						% Nummerierung von Gleichungen links
%fleqn,								% Ausgabe von Gleichungen linksbündig
%draft								% überlangen Zeilen in Ausgabe gekennzeichnet
]
{scrreprt}

%\setcounter{secnumdepth}{3}			% to enable numbering of subsubsections in text
%\setcounter{tocdepth}{1}				% to enable numbering of subsubsections in table of contents: no, we don't want this here

%\clubpenalty10000								% to avoid Hurenkinder and Schusterjungen
%\widowpenalty10000
%\displaywidowpenalty=10000


\usepackage[utf8]{inputenc}			% UTF-8 encoding has to be selectd for all files included!
\usepackage[T1]{fontenc}				% must be included to print ä, ö, ü in bibliography if necessary 											
\usepackage[german,american,KeepShorthandsActive]{babel}
\usepackage{csquotes}

\usepackage{hyphsubst}

\usepackage{url}
\urlstyle{same}

%\usepackage[section]{placeins}	% prevent floating of objects over sections


\usepackage{threeparttable}

\usepackage{lmodern}\normalfont %to load T1lmr.fd 
\DeclareFontShape{T1}{lmr}{bx}{sc} { <-> ssub * cmr/bx/sc }{}

\usepackage{graphicx}						% for inclusion of graphics
	\graphicspath{{images/}{images/kinetic_investigations/}{images/temperature_dependent_measurements/}{images/linear_spectroscopy/}{images/pump_probe_spectroscopy/}{images/theoretical_background/}{images/Gaussian/}}

\usepackage{subcaption}					% several images next to each other


\usepackage[left=2.5cm,right=2.5cm,top=2.0cm,bottom=2.0cm]{geometry}
\addtolength{\footskip}{-.8cm}% Fußbereich 1 cm höher setzen 

\usepackage[format=plain, indention=.2cm, labelfont=bf, textfont=it, font=small, margin=10pt]{caption}
\captionsetup[figure]{skip=10pt}
%\usepackage[font=small,skip=0pt]{caption}

\usepackage[onehalfspacing]{setspace}	% linespacing of onehalf
										
\usepackage{siunitx}
\sisetup{
	detect-mode=true,							% text oder math mode automatically
	locale=DE,										% language
	output-decimal-marker=., 			% decimal separator 
	list-final-separator={ and },	% list seperator
	list-pair-separator={ and }, 	% list seperator
	range-phrase={ to }, 					% range separator
	detect-weight=true,						%
	detect-family=true,						%
	detect-shape=true,						%
}

% defined units %
\DeclareSIUnit{\calorie}{cal}
\DeclareSIUnit{\ppm}{ppm}
\DeclareSIUnit{\eq}{eq}
%

\usepackage{xspace}				% to prevent whitespaces to be swallowed by commands
\usepackage{hyperref}
\pdfstringdefDisableCommands{%
  \def\textsuperscript#1{\textasciicircum(#1)}%
}		


\usepackage{float}												% to prevent floating of objects
\usepackage{booktabs,longtable,multirow}	% for creation of tables
\usepackage[version=4]{mhchem}	
%\usepackage{chemmacros}
	
%references to figures, tables etc.
\usepackage{prettyref}
\newrefformat{eq}{Eq. \textup{\ref{#1}}}
\newrefformat{tab}{Tab. \ref{#1}}
\newrefformat{fig}{Fig. \ref{#1}}
\newrefformat{sec}{section~\ref{#1}} 
\newrefformat{cha}{Chapter~\ref{#1}} 
%-------------------------------------	
	
\usepackage[backend=bibtex,style=chem-acs]{biblatex}
\makeatletter
\def\blx@maxline{77}
\makeatother

		\addbibresource{N:/Unruh/Master/literature/literature.bib}
		\addbibresource{N:/Unruh/Master/literature/PBEh-3c/PBEh-3c.bib}
		\addbibresource{N:/Unruh/Master/literature/ORCA_def2-TZVP/ORCA_def2-TZVP.bib}
		
\include{shortcuts}


\title{tresspec toolkit}
\subtitle{A python package for fast and convenient processing and analysis of time-resolved spectroscopic data}
\date{\today}
\author{Tobias Unruh}



\begin{document}
\maketitle


\begin{abstract}

\end{abstract}
Text

\chapter{Scope of this package}
Over time, the experienced user of the setups for conducting time-resolved spectroscopic measurements faces several reoccuring tasks. This led us to the point to decide to compile a python package that is intended to facilitate and speed up the analysis of time-resolved spectroscopic measurements by comprising convenient tools for the conduction of routine operations on the data sets.

\chapter{Functions Overview}

\section{loaddata}
\subsection{uv-pump-mir-probe}
 

\texttt{   """
    invoke uv\_pump\_mir_probe to load data recorded on the UV-pump-mIR-probe experiment

    :param wrkpath:                     the path containing the *sve.csv files as received from the spectrometer
    :param discard_stitching_blocks:    a list containing the indices of bad stitching blocks which are to be discarded



    :return:
        - runs:                         a list containing the individual measurements
        - runs_sb:                      a list of the individual measurements separated into their stitching blocks
    """
}
%\newcommand{\leadingzero}[1]{\ifnum #1<10 0\the#1\else\the#1\fi}             %%fügt bei 1 bis 9 eine führende Null an
%\newcommand{\todayIV}{\leadingzero{\day}.\leadingzero{\month}.\the\year}     % DD.MM.YYYY

%-------------------------------
\texttt{This is a test phrase.}
%\pagenumbering{Roman}
%	\include{titlepage}
%	\blankpage

%	\include{zeitraum}
%	\blankpage
	
%	\include{eidesstattliche_erklaerung}
%	\blankpage
	
%	\include{acknowledgement}
%	\blankpage

%	\tableofcontents 
%	\blankpage

%-------------------------
%\pagenumbering{arabic}
%\include{introduction}
%\include{theoretical_background}
%\include{synthesis}
%\include{computational_studies}
%\include{linear_spectroscopy}
%\include{cis_trans_isomerization}
%\include{pump_probe_spectroscopy}
%\include{conclusion_and_outlook}


%\printbibliography[heading=bibnumbered]

\end{document}